\section{Further work following progression}
\protect\label{section:worktocome}

The following is the programme of work I need to undertake in order to regard
the project as complete.

In a lot of cases, the work already done will need revisiting and refining and
``iterating until convergence'' achieved.

I see the following as necessary, not in any particular order.

\begin{enumerate}
  \item Better handle bad or unreliable pixels identifying ones prone to
  ``spikes'' and bouts of insensitivity and improving the interpolation over
  them. Where they land on an object, it would be better to interpolate using
  any fitted Gaussian profile.
  \item Optimise as many as possible of the reference object apertures and
  identify irregularly shaped and variable objects.
  \item Look at other filters, such as the \gfilter, possibly the other filters,
  although the number of available reference objects looks poor and the
  {\zfilter} images all have artefacts in them.
  \item Repeat the work done for {\ross} with {\prox} and {\bstar} and consider
  periodograms for them.
  \item Extend the work done to date to the REMIR near infrared images.
\end{enumerate}

\section{Accuracy}
\protect\label{section:accuracy}

It seemed worth looking at whether any adjustments of the exposure times might
improve the accuracy, as many of the objects which might be used for reference
give an ADU count only just above the sky level. The exposure times currently
are 10 seconds for {\prox} and {\ross} and 5 seconds for \bstar.

Table \ref{table:pixvals} is set out the maximum value in any of the observation
files for each target and each filter and the percentage of observation files in
which at least one pixel exceeds a give number of ADUs (this does not currently
take account of defective pixels or cosmics, but these would have negligible
effects on the results).

It is clear that some increase in exposure time, possibly up to 5 times or more
in the case of the \gfilter, would benefit the contrast, except
for the {\ifilter} results, which are generally too close to saturation.
The nature of the instrument, however, may not permit any different exposure
times to be made with different filters.

\begin{table}[!htbp]
\begin{center}
\begin{tabular}{|l|r|rrrr|} \hline
Filter & Max value & \% over 30,000 & \% over 40,000 & \% over 50,000 & \% over
60,000 \\\hline
 \multicolumn{6}{|c|}{\textbf{\bstar} exp 5 sec} \\\hline
\texttt{g} & 63,422 & 0.05 & 0.05 & 0.05 & 0.05 \\
\texttt{i} & 63,448 & 78.94 & 61.15 & 45.71 & 0.77 \\
\texttt{r} & 54,806 & 16.68 & 3.73 & 0.97 & 0.00 \\
\texttt{z} & 63,387 & 12.52 & 3.37 & 0.87 & 0.05 \\
\hline
\multicolumn{6}{|c|}{\textbf{\prox} exp 10 sec} \\\hline
\texttt{g} & 63,466 & 0.38 & 0.30 & 0.25 & 0.18 \\
\texttt{i} & 63,457 & 74.15 & 52.88 & 36.84 & 4.03 \\
\texttt{r} & 63,425 & 0.60 & 0.10 & 0.09 & 0.08 \\
\texttt{z} & 63,456 & 25.38 & 11.64 & 5.47 & 0.16 \\
\hline
\multicolumn{6}{|c|}{\textbf{Ross 154} exp 10 sec} \\\hline
\texttt{g} & 63,453 & 0.41 & 0.35 & 0.28 & 0.22 \\
\texttt{i} & 63,452 & 76.71 & 57.35 & 39.25 & 2.16 \\
\texttt{r} & 63,268 & 7.99 & 2.38 & 0.93 & 0.11 \\
\texttt{z} & 63,447 & 8.33 & 2.14 & 0.45 & 0.04 \\
\hline
\hline
\end{tabular}
\end{center}
\caption{This figures shows the maximum ADU values of any pixel in any of the
observation files for each of the 3 {\rdwarf} targets for each filter. The
following columns show the percentage of pixels in each case which are above
30,000, 40,000, 50,000 and 60,000.}.
\protect\label{table:pixvals}
\end{table}
\clearpage
